% Options for packages loaded elsewhere
\PassOptionsToPackage{unicode}{hyperref}
\PassOptionsToPackage{hyphens}{url}
%
\documentclass[
]{article}
\usepackage{amsmath,amssymb}
\usepackage{iftex}
\ifPDFTeX
  \usepackage[T1]{fontenc}
  \usepackage[utf8]{inputenc}
  \usepackage{textcomp} % provide euro and other symbols
\else % if luatex or xetex
  \usepackage{unicode-math} % this also loads fontspec
  \defaultfontfeatures{Scale=MatchLowercase}
  \defaultfontfeatures[\rmfamily]{Ligatures=TeX,Scale=1}
\fi
\usepackage{lmodern}
\ifPDFTeX\else
  % xetex/luatex font selection
\fi
% Use upquote if available, for straight quotes in verbatim environments
\IfFileExists{upquote.sty}{\usepackage{upquote}}{}
\IfFileExists{microtype.sty}{% use microtype if available
  \usepackage[]{microtype}
  \UseMicrotypeSet[protrusion]{basicmath} % disable protrusion for tt fonts
}{}
\makeatletter
\@ifundefined{KOMAClassName}{% if non-KOMA class
  \IfFileExists{parskip.sty}{%
    \usepackage{parskip}
  }{% else
    \setlength{\parindent}{0pt}
    \setlength{\parskip}{6pt plus 2pt minus 1pt}}
}{% if KOMA class
  \KOMAoptions{parskip=half}}
\makeatother
\usepackage{xcolor}
\usepackage[margin=1in]{geometry}
\usepackage{color}
\usepackage{fancyvrb}
\newcommand{\VerbBar}{|}
\newcommand{\VERB}{\Verb[commandchars=\\\{\}]}
\DefineVerbatimEnvironment{Highlighting}{Verbatim}{commandchars=\\\{\}}
% Add ',fontsize=\small' for more characters per line
\usepackage{framed}
\definecolor{shadecolor}{RGB}{248,248,248}
\newenvironment{Shaded}{\begin{snugshade}}{\end{snugshade}}
\newcommand{\AlertTok}[1]{\textcolor[rgb]{0.94,0.16,0.16}{#1}}
\newcommand{\AnnotationTok}[1]{\textcolor[rgb]{0.56,0.35,0.01}{\textbf{\textit{#1}}}}
\newcommand{\AttributeTok}[1]{\textcolor[rgb]{0.13,0.29,0.53}{#1}}
\newcommand{\BaseNTok}[1]{\textcolor[rgb]{0.00,0.00,0.81}{#1}}
\newcommand{\BuiltInTok}[1]{#1}
\newcommand{\CharTok}[1]{\textcolor[rgb]{0.31,0.60,0.02}{#1}}
\newcommand{\CommentTok}[1]{\textcolor[rgb]{0.56,0.35,0.01}{\textit{#1}}}
\newcommand{\CommentVarTok}[1]{\textcolor[rgb]{0.56,0.35,0.01}{\textbf{\textit{#1}}}}
\newcommand{\ConstantTok}[1]{\textcolor[rgb]{0.56,0.35,0.01}{#1}}
\newcommand{\ControlFlowTok}[1]{\textcolor[rgb]{0.13,0.29,0.53}{\textbf{#1}}}
\newcommand{\DataTypeTok}[1]{\textcolor[rgb]{0.13,0.29,0.53}{#1}}
\newcommand{\DecValTok}[1]{\textcolor[rgb]{0.00,0.00,0.81}{#1}}
\newcommand{\DocumentationTok}[1]{\textcolor[rgb]{0.56,0.35,0.01}{\textbf{\textit{#1}}}}
\newcommand{\ErrorTok}[1]{\textcolor[rgb]{0.64,0.00,0.00}{\textbf{#1}}}
\newcommand{\ExtensionTok}[1]{#1}
\newcommand{\FloatTok}[1]{\textcolor[rgb]{0.00,0.00,0.81}{#1}}
\newcommand{\FunctionTok}[1]{\textcolor[rgb]{0.13,0.29,0.53}{\textbf{#1}}}
\newcommand{\ImportTok}[1]{#1}
\newcommand{\InformationTok}[1]{\textcolor[rgb]{0.56,0.35,0.01}{\textbf{\textit{#1}}}}
\newcommand{\KeywordTok}[1]{\textcolor[rgb]{0.13,0.29,0.53}{\textbf{#1}}}
\newcommand{\NormalTok}[1]{#1}
\newcommand{\OperatorTok}[1]{\textcolor[rgb]{0.81,0.36,0.00}{\textbf{#1}}}
\newcommand{\OtherTok}[1]{\textcolor[rgb]{0.56,0.35,0.01}{#1}}
\newcommand{\PreprocessorTok}[1]{\textcolor[rgb]{0.56,0.35,0.01}{\textit{#1}}}
\newcommand{\RegionMarkerTok}[1]{#1}
\newcommand{\SpecialCharTok}[1]{\textcolor[rgb]{0.81,0.36,0.00}{\textbf{#1}}}
\newcommand{\SpecialStringTok}[1]{\textcolor[rgb]{0.31,0.60,0.02}{#1}}
\newcommand{\StringTok}[1]{\textcolor[rgb]{0.31,0.60,0.02}{#1}}
\newcommand{\VariableTok}[1]{\textcolor[rgb]{0.00,0.00,0.00}{#1}}
\newcommand{\VerbatimStringTok}[1]{\textcolor[rgb]{0.31,0.60,0.02}{#1}}
\newcommand{\WarningTok}[1]{\textcolor[rgb]{0.56,0.35,0.01}{\textbf{\textit{#1}}}}
\usepackage{graphicx}
\makeatletter
\def\maxwidth{\ifdim\Gin@nat@width>\linewidth\linewidth\else\Gin@nat@width\fi}
\def\maxheight{\ifdim\Gin@nat@height>\textheight\textheight\else\Gin@nat@height\fi}
\makeatother
% Scale images if necessary, so that they will not overflow the page
% margins by default, and it is still possible to overwrite the defaults
% using explicit options in \includegraphics[width, height, ...]{}
\setkeys{Gin}{width=\maxwidth,height=\maxheight,keepaspectratio}
% Set default figure placement to htbp
\makeatletter
\def\fps@figure{htbp}
\makeatother
\setlength{\emergencystretch}{3em} % prevent overfull lines
\providecommand{\tightlist}{%
  \setlength{\itemsep}{0pt}\setlength{\parskip}{0pt}}
\setcounter{secnumdepth}{-\maxdimen} % remove section numbering
\ifLuaTeX
  \usepackage{selnolig}  % disable illegal ligatures
\fi
\IfFileExists{bookmark.sty}{\usepackage{bookmark}}{\usepackage{hyperref}}
\IfFileExists{xurl.sty}{\usepackage{xurl}}{} % add URL line breaks if available
\urlstyle{same}
\hypersetup{
  pdftitle={R Notebook},
  hidelinks,
  pdfcreator={LaTeX via pandoc}}

\title{R Notebook}
\author{}
\date{\vspace{-2.5em}}

\begin{document}
\maketitle

\hypertarget{importing-libraries-and-dataset}{%
\section{Importing Libraries and
Dataset}\label{importing-libraries-and-dataset}}

\begin{Shaded}
\begin{Highlighting}[]
\FunctionTok{library}\NormalTok{(tidyverse)}
\end{Highlighting}
\end{Shaded}

\begin{verbatim}
## -- Attaching core tidyverse packages ------------------------ tidyverse 2.0.0 --
## v dplyr     1.1.2     v readr     2.1.4
## v forcats   1.0.0     v stringr   1.5.0
## v ggplot2   3.4.3     v tibble    3.2.1
## v lubridate 1.9.2     v tidyr     1.3.0
## v purrr     1.0.2     
## -- Conflicts ------------------------------------------ tidyverse_conflicts() --
## x dplyr::filter() masks stats::filter()
## x dplyr::lag()    masks stats::lag()
## i Use the conflicted package (<http://conflicted.r-lib.org/>) to force all conflicts to become errors
\end{verbatim}

\begin{Shaded}
\begin{Highlighting}[]
\NormalTok{AutoMobiles}\OtherTok{\textless{}{-}}\FunctionTok{read.csv}\NormalTok{(}\StringTok{"C:/Users/saic3/CSE3046{-}F2{-}LAB\_SLOT\_L3+L4/Datasets/Automobile.csv"}\NormalTok{)}
\end{Highlighting}
\end{Shaded}

\#Check For Null Values:

\begin{Shaded}
\begin{Highlighting}[]
\NormalTok{ToNull}\OtherTok{\textless{}{-}}\FunctionTok{is.na}\NormalTok{(AutoMobiles}\SpecialCharTok{$}\NormalTok{price)}
\FunctionTok{sum}\NormalTok{(ToNull)}
\end{Highlighting}
\end{Shaded}

\begin{verbatim}
## [1] 3
\end{verbatim}

\hypertarget{removing-the-na-values}{%
\section{Removing the NA Values}\label{removing-the-na-values}}

\begin{Shaded}
\begin{Highlighting}[]
\NormalTok{AutoMobiles}\OtherTok{\textless{}{-}}\NormalTok{AutoMobiles}\SpecialCharTok{\%\textgreater{}\%}\FunctionTok{fill}\NormalTok{(price,}\AttributeTok{.direction =} \StringTok{"down"}\NormalTok{)}
\FunctionTok{sum}\NormalTok{(}\FunctionTok{is.na}\NormalTok{(AutoMobiles}\SpecialCharTok{$}\NormalTok{price))}
\end{Highlighting}
\end{Shaded}

\begin{verbatim}
## [1] 0
\end{verbatim}

\hypertarget{question-1find-last-n-rows-with-price-14000}{%
\section{Question-1:Find last n rows with price \textgreater{}
14000?}\label{question-1find-last-n-rows-with-price-14000}}

\begin{Shaded}
\begin{Highlighting}[]
\FunctionTok{tail}\NormalTok{(AutoMobiles[}\StringTok{"price"}\SpecialCharTok{\textgreater{}}\DecValTok{14000}\NormalTok{])}
\end{Highlighting}
\end{Shaded}

\begin{verbatim}
##    index    company body.style wheel.base length engine.type num.of.cylinders
## 56    80 volkswagen      sedan       97.3  171.7         ohc             four
## 57    81 volkswagen      sedan       97.3  171.7         ohc             four
## 58    82 volkswagen      sedan       97.3  171.7         ohc             four
## 59    86 volkswagen      sedan       97.3  171.7         ohc             four
## 60    87      volvo      sedan      104.3  188.8         ohc             four
## 61    88      volvo      wagon      104.3  188.8         ohc             four
##    horsepower average.mileage price
## 56         52              37  7775
## 57         85              27  7975
## 58         52              37  7995
## 59        100              26  9995
## 60        114              23 12940
## 61        114              23 13415
\end{verbatim}

\hypertarget{question-2-count-the-total-company-in-the-dataset}{%
\section{Question-2: Count the total company in the
dataset?}\label{question-2-count-the-total-company-in-the-dataset}}

\begin{Shaded}
\begin{Highlighting}[]
\FunctionTok{unique}\NormalTok{(AutoMobiles}\SpecialCharTok{$}\NormalTok{company)}\SpecialCharTok{\%\textgreater{}\%}\FunctionTok{length}\NormalTok{()}
\end{Highlighting}
\end{Shaded}

\begin{verbatim}
## [1] 16
\end{verbatim}

\begin{Shaded}
\begin{Highlighting}[]
\FunctionTok{unique}\NormalTok{(AutoMobiles}\SpecialCharTok{$}\NormalTok{company)}
\end{Highlighting}
\end{Shaded}

\begin{verbatim}
##  [1] "alfa-romero"   "audi"          "bmw"           "chevrolet"    
##  [5] "dodge"         "honda"         "isuzu"         "jaguar"       
##  [9] "mazda"         "mercedes-benz" "mitsubishi"    "nissan"       
## [13] "porsche"       "toyota"        "volkswagen"    "volvo"
\end{verbatim}

\hypertarget{question-3-find-all-the-rows-with-body-style-sedan}{%
\section{Question-3: Find all the rows with body-style
sedan?}\label{question-3-find-all-the-rows-with-body-style-sedan}}

\begin{Shaded}
\begin{Highlighting}[]
\NormalTok{AutoMobiles}\SpecialCharTok{\%\textgreater{}\%}\FunctionTok{filter}\NormalTok{(body.style}\SpecialCharTok{==}\StringTok{"sedan"}\NormalTok{)}
\end{Highlighting}
\end{Shaded}

\begin{verbatim}
##    index       company body.style wheel.base length engine.type
## 1      3          audi      sedan       99.8  176.6         ohc
## 2      4          audi      sedan       99.4  176.6         ohc
## 3      5          audi      sedan       99.8  177.3         ohc
## 4      9           bmw      sedan      101.2  176.8         ohc
## 5     10           bmw      sedan      101.2  176.8         ohc
## 6     11           bmw      sedan      101.2  176.8         ohc
## 7     13           bmw      sedan      103.5  189.0         ohc
## 8     14           bmw      sedan      103.5  193.8         ohc
## 9     15           bmw      sedan      110.0  197.0         ohc
## 10    18     chevrolet      sedan       94.5  158.8         ohc
## 11    28         honda      sedan       96.5  175.4         ohc
## 12    29         honda      sedan       96.5  169.1         ohc
## 13    30         isuzu      sedan       94.3  170.7         ohc
## 14    31         isuzu      sedan       94.5  155.9         ohc
## 15    32         isuzu      sedan       94.5  155.9         ohc
## 16    33        jaguar      sedan      113.0  199.6        dohc
## 17    34        jaguar      sedan      113.0  199.6        dohc
## 18    35        jaguar      sedan      102.0  191.7        ohcv
## 19    43         mazda      sedan      104.9  175.0         ohc
## 20    44 mercedes-benz      sedan      110.0  190.9         ohc
## 21    46 mercedes-benz      sedan      120.9  208.1        ohcv
## 22    51    mitsubishi      sedan       96.3  172.4         ohc
## 23    52    mitsubishi      sedan       96.3  172.4         ohc
## 24    53        nissan      sedan       94.5  165.3         ohc
## 25    54        nissan      sedan       94.5  165.3         ohc
## 26    55        nissan      sedan       94.5  165.3         ohc
## 27    57        nissan      sedan      100.4  184.6        ohcv
## 28    80    volkswagen      sedan       97.3  171.7         ohc
## 29    81    volkswagen      sedan       97.3  171.7         ohc
## 30    82    volkswagen      sedan       97.3  171.7         ohc
## 31    86    volkswagen      sedan       97.3  171.7         ohc
## 32    87         volvo      sedan      104.3  188.8         ohc
##    num.of.cylinders horsepower average.mileage price
## 1              four        102              24 13950
## 2              five        115              18 17450
## 3              five        110              19 15250
## 4              four        101              23 16430
## 5              four        101              23 16925
## 6               six        121              21 20970
## 7               six        182              16 30760
## 8               six        182              16 41315
## 9               six        182              15 36880
## 10             four         70              38  6575
## 11             four        101              24 12945
## 12             four        100              25 10345
## 13             four         78              24  6785
## 14             four         70              38  6785
## 15             four         70              38  6785
## 16              six        176              15 32250
## 17              six        176              15 35550
## 18           twelve        262              13 36000
## 19             four         72              31 18344
## 20             five        123              22 25552
## 21            eight        184              14 40960
## 22             four         88              25  6989
## 23             four         88              25  8189
## 24             four         55              45  7099
## 25             four         69              31  6649
## 26             four         69              31  6849
## 27              six        152              19 13499
## 28             four         52              37  7775
## 29             four         85              27  7975
## 30             four         52              37  7995
## 31             four        100              26  9995
## 32             four        114              23 12940
\end{verbatim}

\begin{Shaded}
\begin{Highlighting}[]
\NormalTok{AutoMobiles}\SpecialCharTok{\%\textgreater{}\%}\FunctionTok{filter}\NormalTok{(body.style}\SpecialCharTok{==}\StringTok{"sedan"}\NormalTok{)}\SpecialCharTok{\%\textgreater{}\%}\FunctionTok{count}\NormalTok{()}
\end{Highlighting}
\end{Shaded}

\begin{verbatim}
##    n
## 1 32
\end{verbatim}

\hypertarget{question-4-find-the-3rd-most-expensive-car-price-and-company-name.}{%
\section{Question-4: Find the 3rd most expensive car price and company
name.}\label{question-4-find-the-3rd-most-expensive-car-price-and-company-name.}}

\begin{Shaded}
\begin{Highlighting}[]
\NormalTok{Q4}\OtherTok{\textless{}{-}}\NormalTok{AutoMobiles}\SpecialCharTok{\%\textgreater{}\%}\FunctionTok{group\_by}\NormalTok{(company)}\SpecialCharTok{\%\textgreater{}\%}\FunctionTok{summarise}\NormalTok{(}\AttributeTok{newPrice=}\FunctionTok{max}\NormalTok{(price))}\SpecialCharTok{\%\textgreater{}\%}\FunctionTok{arrange}\NormalTok{(}\FunctionTok{desc}\NormalTok{(newPrice))}
\CommentTok{\#Since Due to 0{-}based Indexing and No of Companies are 16 from the Q2 We Use 16{-}2 as Index}
\NormalTok{Q4[}\DecValTok{3}\NormalTok{,]}
\end{Highlighting}
\end{Shaded}

\begin{verbatim}
## # A tibble: 1 x 2
##   company newPrice
##   <chr>      <int>
## 1 porsche    37028
\end{verbatim}

\#Question-5: Find the most expensive car for each company. \#Similar to
Question 4 Where Finding the Third Highest Car Price and Its Company

\begin{Shaded}
\begin{Highlighting}[]
\NormalTok{Q4}
\end{Highlighting}
\end{Shaded}

\begin{verbatim}
## # A tibble: 16 x 2
##    company       newPrice
##    <chr>            <int>
##  1 mercedes-benz    45400
##  2 bmw              41315
##  3 porsche          37028
##  4 jaguar           36000
##  5 audi             18920
##  6 mazda            18344
##  7 alfa-romero      16500
##  8 toyota           15750
##  9 nissan           13499
## 10 volvo            13415
## 11 honda            12945
## 12 volkswagen        9995
## 13 mitsubishi        8189
## 14 isuzu             6785
## 15 chevrolet         6575
## 16 dodge             6377
\end{verbatim}

\#Question 6: Print all Toyota cars details

\begin{Shaded}
\begin{Highlighting}[]
\NormalTok{AutoMobiles}\SpecialCharTok{\%\textgreater{}\%}\FunctionTok{filter}\NormalTok{(company}\SpecialCharTok{==}\StringTok{"toyota"}\NormalTok{)}
\end{Highlighting}
\end{Shaded}

\begin{verbatim}
##   index company body.style wheel.base length engine.type num.of.cylinders
## 1    66  toyota  hatchback       95.7  158.7         ohc             four
## 2    67  toyota  hatchback       95.7  158.7         ohc             four
## 3    68  toyota  hatchback       95.7  158.7         ohc             four
## 4    69  toyota      wagon       95.7  169.7         ohc             four
## 5    70  toyota      wagon       95.7  169.7         ohc             four
## 6    71  toyota      wagon       95.7  169.7         ohc             four
## 7    79  toyota      wagon      104.5  187.8        dohc              six
##   horsepower average.mileage price
## 1         62              35  5348
## 2         62              31  6338
## 3         62              31  6488
## 4         62              31  6918
## 5         62              27  7898
## 6         62              27  8778
## 7        156              19 15750
\end{verbatim}

\hypertarget{question-7find-the-count-of-convertible-type-cars-in-alfa-romero-company}{%
\section{Question-7:Find the count of ``convertible'' type cars in
``alfa-romero''
company}\label{question-7find-the-count-of-convertible-type-cars-in-alfa-romero-company}}

\begin{Shaded}
\begin{Highlighting}[]
\FunctionTok{count}\NormalTok{(AutoMobiles}\SpecialCharTok{\%\textgreater{}\%}\FunctionTok{filter}\NormalTok{(company}\SpecialCharTok{==}\StringTok{"alfa{-}romero"}\SpecialCharTok{\&}\NormalTok{body.style}\SpecialCharTok{==}\StringTok{"convertible"}\NormalTok{))}
\end{Highlighting}
\end{Shaded}

\begin{verbatim}
##   n
## 1 2
\end{verbatim}

\begin{Shaded}
\begin{Highlighting}[]
\NormalTok{AutoMobiles}\SpecialCharTok{\%\textgreater{}\%}\FunctionTok{filter}\NormalTok{(company}\SpecialCharTok{==}\StringTok{"alfa{-}romero"}\SpecialCharTok{\&}\NormalTok{body.style}\SpecialCharTok{==}\StringTok{"convertible"}\NormalTok{)}
\end{Highlighting}
\end{Shaded}

\begin{verbatim}
##   index     company  body.style wheel.base length engine.type num.of.cylinders
## 1     0 alfa-romero convertible       88.6  168.8        dohc             four
## 2     1 alfa-romero convertible       88.6  168.8        dohc             four
##   horsepower average.mileage price
## 1        111              21 13495
## 2        111              21 16500
\end{verbatim}

\hypertarget{question-8-create-a-vector-with-20-numeric-items-and-extract-top-2-most-frequent-items-of-a-vector}{%
\section{Question-8: Create a vector with 20 numeric items and extract
top 2 most frequent items of a
vector?}\label{question-8-create-a-vector-with-20-numeric-items-and-extract-top-2-most-frequent-items-of-a-vector}}

\begin{Shaded}
\begin{Highlighting}[]
\NormalTok{numeric\_vector}\OtherTok{\textless{}{-}}\FunctionTok{c}\NormalTok{(}\DecValTok{1}\NormalTok{, }\DecValTok{2}\NormalTok{, }\DecValTok{3}\NormalTok{, }\DecValTok{2}\NormalTok{, }\DecValTok{1}\NormalTok{, }\DecValTok{4}\NormalTok{, }\DecValTok{5}\NormalTok{, }\DecValTok{1}\NormalTok{, }\DecValTok{2}\NormalTok{, }\DecValTok{6}\NormalTok{, }\DecValTok{7}\NormalTok{, }\DecValTok{7}\NormalTok{, }\DecValTok{8}\NormalTok{, }\DecValTok{9}\NormalTok{, }\DecValTok{3}\NormalTok{, }\DecValTok{10}\NormalTok{, }\DecValTok{5}\NormalTok{, }\DecValTok{5}\NormalTok{, }\DecValTok{2}\NormalTok{, }\DecValTok{1}\NormalTok{)}
\NormalTok{item\_freq }\OtherTok{\textless{}{-}} \FunctionTok{table}\NormalTok{(numeric\_vector)}
\CommentTok{\# Sort the frequencies in decreasing order}
\NormalTok{sorted\_freq }\OtherTok{\textless{}{-}} \FunctionTok{sort}\NormalTok{(item\_freq, }\AttributeTok{decreasing =} \ConstantTok{TRUE}\NormalTok{)}
\CommentTok{\# Extract the top 2 most frequent items}
\NormalTok{top\_2\_items }\OtherTok{\textless{}{-}} \FunctionTok{as.numeric}\NormalTok{(}\FunctionTok{names}\NormalTok{(sorted\_freq[}\DecValTok{1}\SpecialCharTok{:}\DecValTok{2}\NormalTok{]))}

\FunctionTok{print}\NormalTok{(top\_2\_items)}
\end{Highlighting}
\end{Shaded}

\begin{verbatim}
## [1] 1 2
\end{verbatim}

\#Question-9:Create two dataframe with different attributes and merge
them column wise.

\begin{Shaded}
\begin{Highlighting}[]
\NormalTok{df1}\OtherTok{\textless{}{-}}\FunctionTok{data.frame}\NormalTok{(}\AttributeTok{Students=}\FunctionTok{c}\NormalTok{(}\StringTok{"Somu"}\NormalTok{,}\StringTok{"Venu"}\NormalTok{,}\StringTok{"Venkat"}\NormalTok{,}\StringTok{"Sri"}\NormalTok{,}\StringTok{"Charvi"}\NormalTok{,}\StringTok{"Marky"}\NormalTok{,}\StringTok{"Duplex"}\NormalTok{),}\AttributeTok{Marks=}\FunctionTok{c}\NormalTok{(}\DecValTok{10}\NormalTok{,}\DecValTok{100}\NormalTok{,}\DecValTok{49}\NormalTok{,}\DecValTok{40}\NormalTok{,}\DecValTok{15}\NormalTok{,}\DecValTok{21}\NormalTok{,}\DecValTok{95}\NormalTok{))}
\NormalTok{df1}
\end{Highlighting}
\end{Shaded}

\begin{verbatim}
##   Students Marks
## 1     Somu    10
## 2     Venu   100
## 3   Venkat    49
## 4      Sri    40
## 5   Charvi    15
## 6    Marky    21
## 7   Duplex    95
\end{verbatim}

\begin{Shaded}
\begin{Highlighting}[]
\NormalTok{df2}\OtherTok{\textless{}{-}}\FunctionTok{data.frame}\NormalTok{(}\AttributeTok{Items=}\FunctionTok{c}\NormalTok{(}\StringTok{"Rice"}\NormalTok{,}\StringTok{"Cofee"}\NormalTok{,}\StringTok{"Tea"}\NormalTok{,}\StringTok{"Oil"}\NormalTok{,}\StringTok{"Vegetables"}\NormalTok{,}\StringTok{"Fruits"}\NormalTok{,}\StringTok{"Ghee"}\NormalTok{),}\AttributeTok{Prices=}\FunctionTok{c}\NormalTok{(}\DecValTok{12}\NormalTok{,}\DecValTok{80}\NormalTok{,}\DecValTok{29}\NormalTok{,}\DecValTok{87}\NormalTok{,}\DecValTok{2}\NormalTok{,}\DecValTok{13}\NormalTok{,}\DecValTok{16}\NormalTok{))}
\NormalTok{df2}
\end{Highlighting}
\end{Shaded}

\begin{verbatim}
##        Items Prices
## 1       Rice     12
## 2      Cofee     80
## 3        Tea     29
## 4        Oil     87
## 5 Vegetables      2
## 6     Fruits     13
## 7       Ghee     16
\end{verbatim}

\begin{Shaded}
\begin{Highlighting}[]
\NormalTok{df1}\OtherTok{\textless{}{-}}\FunctionTok{cbind}\NormalTok{(df1,df2)}
\NormalTok{df1}
\end{Highlighting}
\end{Shaded}

\begin{verbatim}
##   Students Marks      Items Prices
## 1     Somu    10       Rice     12
## 2     Venu   100      Cofee     80
## 3   Venkat    49        Tea     29
## 4      Sri    40        Oil     87
## 5   Charvi    15 Vegetables      2
## 6    Marky    21     Fruits     13
## 7   Duplex    95       Ghee     16
\end{verbatim}

\#QUestion-10: Create two dataframe with the same attributes and merge
them row wise.

\begin{Shaded}
\begin{Highlighting}[]
\NormalTok{df1}\OtherTok{\textless{}{-}}\FunctionTok{data.frame}\NormalTok{(}\AttributeTok{Students=}\FunctionTok{c}\NormalTok{(}\StringTok{"Somu"}\NormalTok{,}\StringTok{"Venu"}\NormalTok{,}\StringTok{"Venkat"}\NormalTok{,}\StringTok{"Sri"}\NormalTok{,}\StringTok{"Charvi"}\NormalTok{,}\StringTok{"Marky"}\NormalTok{,}\StringTok{"Duplex"}\NormalTok{),}\AttributeTok{Marks=}\FunctionTok{c}\NormalTok{(}\DecValTok{10}\NormalTok{,}\DecValTok{100}\NormalTok{,}\DecValTok{49}\NormalTok{,}\DecValTok{40}\NormalTok{,}\DecValTok{15}\NormalTok{,}\DecValTok{21}\NormalTok{,}\DecValTok{95}\NormalTok{))}
\NormalTok{df3}\OtherTok{\textless{}{-}}\FunctionTok{data.frame}\NormalTok{(}\AttributeTok{Students=}\FunctionTok{c}\NormalTok{(}\StringTok{"Sai"}\NormalTok{,}\StringTok{"Kent"}\NormalTok{,}\StringTok{"Bruce"}\NormalTok{),}\AttributeTok{Marks=}\FunctionTok{c}\NormalTok{(}\DecValTok{1000}\NormalTok{,}\DecValTok{1}\NormalTok{,}\DecValTok{99}\NormalTok{))}
\NormalTok{df1}\OtherTok{\textless{}{-}}\FunctionTok{rbind}\NormalTok{(df1,df3)}
\NormalTok{df1}
\end{Highlighting}
\end{Shaded}

\begin{verbatim}
##    Students Marks
## 1      Somu    10
## 2      Venu   100
## 3    Venkat    49
## 4       Sri    40
## 5    Charvi    15
## 6     Marky    21
## 7    Duplex    95
## 8       Sai  1000
## 9      Kent     1
## 10    Bruce    99
\end{verbatim}

\end{document}
